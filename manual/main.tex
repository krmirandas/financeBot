\documentclass{article}

\usepackage[spanish]{babel}
\usepackage[utf8]{inputenc}
\usepackage{color}
\usepackage{graphicx}
\usepackage{hyperref}
\usepackage{listings}
\usepackage{ragged2e}
\usepackage{wrapfig}

\pagenumbering{Roman}
\definecolor{lightgray}{rgb}{.7,.7,.7}%color
\renewcommand{\lstlistingname}{}%edit label in lstlisting

\hypersetup{
  colorlinks=true,
  linkcolor=blue,
  filecolor=magenta,      
  urlcolor=cyan,
}

\title{\huge Como Instalar Finance Bot 1.0}
\author{Galeana Araujo Emiliano \and Miranda Sánchez Kevin Ricardo}
\date{Mayo 07 2020}

\begin{document}

\maketitle

\newpage

\section{Introduccion}

\justifying
Un chatbot es un sistema computacional cuyo objetivo es sostener una conversación con un usuario humano usando lenguaje natural, es decir el lenguaje que usamos diario a diario cuando hablamos con otras personas.\par

En esta propuesta usamos chatovoice basados en la tecnologia AIML que es un lenguaje de programación para chatbots basado en pattern matching de expresiones regulares y lo dicho por el usuario.

\section{Finance Bot}

\justifying
Finance Bot es un sistema que te ayudar a elegir si te conviene invertir o no en una moneda.

\newpage
\section{Instalación}

Primero es necesario copiar el repositorio donde esta \href{https://github.com/ivanvladimir/chatvoice}{Chatvoice}
Una vez que se clona el repositorio, se debe abrir el Prompt de Anaconda y se debe ir a la ruta del repositorio clonado.\par

Abre el archivo environment.yml y elimine las líneas:
\begin{itemize}
\item cpuonly=1.0
\item pytorch=1.6.0
\item torchvision=0.7.0
\end{itemize}

Para añadir descargar el modulo de financeBot
\begin{lstlisting}[frame=single,language=bash,caption=Crear ambiente]
  $ git submodule init
\end{lstlisting}

O bien

\begin{lstlisting}[language=bash,caption=Crear ambiente]
  $ git clone --recursive https://github.com/krmirandas/financeBot
\end{lstlisting}

Para crear un ambiente:

\begin{lstlisting}[frame=single,language=bash,caption=Crear ambiente]
  conda env create -f environment.yml
\end{lstlisting}

Si todo sale bien, aparecera un mensaje como el siguiente:

\begin{lstlisting}[frame=single,language=bash,caption=Activar ambiente]
  To activate this environment, use

     $ conda activate cv

To deactivate an active environment, use

     $ conda deactivate
\end{lstlisting}

Para verificar que se creó el entorno, verifique la lista de entornos conda con el comando:

\begin{lstlisting}[frame=single,language=bash,caption=Verificar ambiente]
  $ conda info --envs
\end{lstlisting}

¿Dónde podemos ubicar el entorno cv creado que vamos a activar?

\begin{lstlisting}[frame=single,language=bash,caption=Activar ambiente]
  $ conda activate cv 
\end{lstlisting}

Actualicemos el ambiente

\begin{lstlisting}[frame=single,language=bash,caption=Actualizar ambiente]
  $ conda env update -f environment.yml
\end{lstlisting}

\section{Uso}

Para ejecutar el ambiente

\begin{lstlisting}[language=bash,caption=Correr chatbot]
  $ python src/chatvoice.py conversations/financeBot/main.yaml
\end{lstlisting}

\begin{lstlisting}[frame=single,language=bash,caption=Resultado]
  ROBOT: hola
  USER: Jorge
  ROBOT: mucho gusto en conocerte
  USER: Bien!
  ROBOT: Hay algunas cosas que se
  ROBOT: hecho benito juarez nacio en un 21 marzo
  ROBOT: adios Jorge
\end{lstlisting}

Con síntesis de Google
Se tiene que instalar mpg321

\begin{lstlisting}[language=bash,caption=Correr chatbot]
  $ sudo apt install mpg321
\end{lstlisting}

Para correrlo
\begin{lstlisting}[language=bash,caption=Correr chatbot]
  $ python src/chatvoice.py conversations/financeBot/main.yaml --google_tts
\end{lstlisting}

\end{document}
