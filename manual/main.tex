\documentclass{article}

\usepackage[spanish]{babel}
\usepackage[utf8]{inputenc}
\usepackage{color}
\usepackage{graphicx}
\usepackage{hyperref}
\usepackage{listings}
\usepackage{ragged2e}
\usepackage{wrapfig}

\pagenumbering{Roman}
\definecolor{lightgray}{rgb}{.7,.7,.7}%color
\renewcommand{\lstlistingname}{}%edit label in lstlisting

\hypersetup{
  colorlinks=true,
  linkcolor=blue,
  filecolor=magenta,      
  urlcolor=cyan,
}

\title{\huge Como Instalar Finance Bot 1.0}
\author{Galeana Araujo Emiliano \and Miranda Sánchez Kevin Ricardo}
\date{Mayo 07 2020}

\begin{document}

\maketitle

\newpage

\section{Introduccion}

\justifying
Un chatbot es un sistema computacional cuyo objetivo es sostener una conversación con un usuario humano usando lenguaje natural, es decir el lenguaje que usamos diario a diario cuando hablamos con otras personas.\par

En esta propuesta usamos chatovoice basados en la tecnologia AIML que es un lenguaje de programación para chatbots basado en pattern matching de expresiones regulares y lo dicho por el usuario.

\newpage
\section{Instalación}
Primero es necesario copiar el repositorio donde esta \href{https://github.com/ivanvladimir/chatvoice}{Chatvoice}
Una vez que se clona el repositorio, se debe abrir el Prompt de Anaconda y se debe ir a la ruta del repositorio clonado.\par

Abre el archivo environment.yml y elimine las líneas:
\begin{itemize}
\item cpuonly=1.0
\item pytorch=1.6.0
\item torchvision=0.7.0
\end{itemize}

Para crear un ambiente:

\begin{lstlisting}[frame=single,language=bash,caption=etiqueta]
  npm install mocha -g
\end{lstlisting}

% \begin{figure}[h]
%     \includegraphics[scale=0.5]{cp1.png}
%     \caption{Pantalla De Bienvenida}

% \end{figure}
% Una vez que tengamos la pantalla de bienvenida a la vista, el proceso de instalacion habra comenzado, en este caso , debemos elegir el idioma en el que queremos que nuestro sistema operativo y todas las aplicaciones se muestren en pantalla , en este caso elegiremos español , una vez que eligamos un idioma para instalr , auntomaticamente el instalador cambiara de idioma como se muestra en la siguiente imagen.
% \newpage



\end{document}
